\chapter{Probability}

\begin{definition}
	Let $S$ be a set. A $\sigma$-algebra on $S$ is a set $\mathcal{B} \subset \mathcal{P}(S)$ such that 
	
	\begin{enumerate}
		\item 
	\end{enumerate}
\end{definition}
(finish this)

\begin{definition}
	Let $(S, \mathcal{A}, \mu)$ be a measure space. $\mu$ is called a probability measure if $\mu(S) = 1$. If $\mu$ is a probability, this measure space is called a probability space. In probability spaces, the measure is usually denoted $\bbp$ instead of $\mu$.
\end{definition}

If $(S, \mathcal{A}, \mu)$ is a measure space such that $\mu(S) < \infty$, one can easily define an equivalent measure space by $(S, \mathcal{A}, \frac{\mu}{\mu(S)})$, which is easily seen to be a probability space.

\begin{exmp}
	Let $n$ be a positive integer. The standard probability measure in $\bbn_n$ comes from the counting measure. Let $\mathcal{A} = \mathcal{P}$ and $\mu(A) = |A|$ for $A \in \mathcal{P}$. Using the process just described, let 
	
	\begin{center}
		$\bbp(A) = \frac{|A|}{|S|}$.
	\end{center}
	

\end{exmp}

\section{Introduction to Random Walks}

\subsection{Random Walks on Graphs}
Let $X = (X^0, X^1)$ be a locally finite graph. For every $x \in X^0$, consider the standard finite probability distribution on $X^1(x)$. We now define a sequence of random variables $(x_n)_{\bbn}$ recursively by 

\begin{center}
	$$\bbp(x_n = x) = \sum_{y \in X^0} \bbp(x_{n-1} = y`)  \frac{\chi(y, N_X(x))}{|N_X(y)|}   $$ 
\end{center}
(finish)






