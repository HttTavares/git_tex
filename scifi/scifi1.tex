\documentclass[]{article}

%opening
\title{SCI FI}
\author{Vniversvs}

\begin{document}
	
	\maketitle
	
	\begin{abstract}
		
	\end{abstract}
	
	\section{Colonizing Andromeda}
	
	distance 25000 LY
	technology: starlifting.
	
	\section{The First Ring}
	
	This is the story of the first orbital international superstructure, called the first ring. 
	
	The name The First Ring was decided by a committee of public relations from the UN, whose missions was to, first of all, mitigate the high level of pushback by the environmentalists, naysayers and \textit{denialists} and even those who were just scared and did not comprehend well the work underlying the project. It ended up increasing it. It is, however, believed that this negative reaction should fade away in 2 or 3 generations, at which time only the economic and social benefits will remain in culture and memory. The pushback is therefore considered a hiccup in the history of orbital colonization.
	
	It indeed had secondary mission parameters such as improving public opinion, encouragement of teams in organizations involved in its construction, operations and maintenance. Also make a statement about the political and economic commitment of the UN with the sustainable development of all planetary regions.
	
	It also implied at a Second Ring.
	
	The only countries above which the ring extended were Chile and Argentina. It goes around the Earth, for diplomatic reasons avoiding as many countries as it could while maintaining maximum logistic efficiency. Other interested countries built platforms in the south pacific and Indian oceans that made it easier for them to ring launch. So far, 121 countries have built platforms in areas close to the tethers.
	
	These platforms also generated waves of discontent about the problems and issues of oceanic environments, now in much better shape than what is was for most of the XXIth century
	
	
	
	In reality, the ring allowed for much cheaper and easier space and orbital launching, having decreased net environmental impact of spacing by 82\% and orbital launching by 91\% on average, while costs for spacing decreased by 54\% and for orbital launching by 66\%.  
	
	But of course it was not merely about percentages. The historical impact of the First Ring could be summed up as:
	
	\begin{center}
		It allowed for the real beginning of space colonization by making orbital launching a daily happening.  
	\end{center}
	
	Even before it was completed, it had already helped launch 14 space stations for 11 countries, 2 \textbf{UN} colonization missions to mars and the moon, 33 scientific missions to the whole solar system. 
	
	
	
\end{document}
