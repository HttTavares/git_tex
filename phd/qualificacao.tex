%\documentclass[]{article}
\documentclass[14pt,a4paper]{article}
\usepackage{extsizes}
% Using the geometry package with a small
% page size to create the article graphic
%\usepackage[paperheight=11.7in,paperwidth=8.3in,top=15mm,bottom=25mm,left=15mm,right=15mm]{geometry}
\usepackage{lipsum}



%opening
\title{Plano Atualizado de Trabalho}
\author{Henrique Teixeira Tyrrell Tavares}

\begin{document}
	
	\maketitle
	
	\begin{abstract}
		
	\end{abstract}
	
	\section{Introdução}
	
	As consequências sociais, políticas, econômicas e ambientais do avanço técnico-científico da produção ainda são tema de estudo e debate. 
	
	Se por um lado, diversos índices sócio-econômicos e de qualidade de vida melhoraram nos últimos $2$ séculos, há ainda inúmeras necessidades populacionais não satisfeitas, especialmente se levarmos em consideração as questões maiores das desigualdades, sejam elas de gênero, classe, raciais, regionais ou globais. 
	
	Por outro lado, muitos indicadores ambientais começaram a entrar em declínio por causa deste avanço. De fato, já é consenso entre a comunidade científica de estudos do clima e ecologia que a capacidade de interferência da produção econômica no ambiente é tamanho que engendramos uma nova era geológica da terra, o antropoceno (ref).
	
	Neste sentido organizações internacionais e nacionais criaram objetivos concretos, chamados objetivos do desenvolvimento sustentável (ODS) (ref), para guiar os agentes político-econômicos tanto quanto fornecer uma maneira de medir progresso sócio-ambiental, além de outros objetivos. Mais especificamente, os ODSs usam uma delineação a partir de \textit{framework} teórico razoavelmente bem aceito dentro da comunidade científica para entendimento da questão econômico-ambiental enquanto oferecem objetivos concretos. 
	
	Por exemplo, este \textit{framework} teórico formula a necessidade de criação de uma economia sustentável a partir do paradigma da transição sustentável, isto é, uma gradual mudança social levada principalmente pela iniciativa própria dos agentes econômicos, com encorajamento social e governamental. Adicionalmente, neste \textit{framework} se divide a economia em setores, como de praxe nas muitas teorias econômicas, mas com isso divide também o impacto ambiental da do atual modo de produção de acordo com estas divisões.
	
	Seguindo paradigma da transição, pode-se argumentar que houve uma convergência concreta e cultural na necessidade da transição energética, cujo objetivo é transformar a matriz energética em "energia verde". A maior parte dos esforços financeiros, tecno-científicos, operacionais e culturais têm se dado neste setor e provavelmente continuará a ser assim até aproximadamente o final da década, em detrimento de outros setores. 
	
	Por exemplo, dentre os setores da economia global que mais têm impacto negativo nos indicadores ambientais encontramos a agricultura, particularmente no âmbito do desmatamento florestal e consequente perda de biodiversidade, degradação do solo, emissões de gases do efeito estufa e acidificação do oceano devido aos insumos de reposição nutricional do solo, dentre outros (ref). Neste estudo, vamos focar nossos esforços em entender melhor a nova realidade deste setor, enquanto também pretendemos contribuir com o esforço de transição da agricultura à sustentabilidade.

	\subsection{Agricultura}
	
		A agricultura, infelizmente, ainda não é suficientemente estudada enquanto \textit{driver} da crise ambiental, e de forma também insuficiente estuda-se como reverter este cenário, ou seja, não entendemos bem como opera a transição sustentável do modelo de produção agrícola, em termos similares como está acontecendo com o setor energético da economia (ref). 
		
		Como mencionado, sabemos que a agricultura é quase tão danosa quanto a produção de 
		energia em termos de emissões de gases do efeito estufa, sendo responsável por $27\%$ das emissões totais no mundo (ref). Sabemos também serem extramamente problemáticos os agrotóxicos que contribuem para a desertificação, contaminação de aquíferos, perda de biodiversidade e causam problemas de saúde (ref), complementos nutricionais químicos - principalmente baseados em \textbf{NPK} - que também contribuem para a degradação do solo, perda de biodiversidade e ainda para a acidificação do oceano (ref), o uso crescente de energia em unidades produtivas de alta produtividade.
		
		Menos estudados são impactos que têm surgido recentemente com o novo grau de digitalização da economia, como por exemplo a criação de aplicativos de entrega de alimentos e o aumento relativo da emissão de gases do efeito estufa por aumento do delivery, ou a diminuição do uso de recursos devido a melhora do manejo usando tecnologia de sensoreamento remoto e automação de precisão.
		
	\subsection{A Tecnologia}
	
		No caso da agricultura, houve uma grande mudança de modelo produtivo nos meados do século \textbf{XX} hoje conhecido como revolução verde. A forma anterior de produção agrícola, baseada na propriedade familiar de pequeno e médio porte, técnicas tradicionais, pluricultura, baixa intensidade foi substituída nos países desenvolvidos por um modelo baseado na grande propriedade privada, técnicas indústriais bioquímicas e mecânicas, monocultura e alta intensidade. No Brasil, entretanto, a mudança seguiu um padrão diferente: nunca foi dominante a pluricultura ou a propriedade familiar da terra, e apenas recentemente foram introduzidas as técnicas agro-industriais. Essa mudança resultou no aumento da produtividade e resolução do dilema malthusiano, porém também resultou no aumento do impacto ambiental negativo deste setor.
		
		De fato, muitos atores sociais e econômicos apontam para soluções técnicas ou tecnológica para esta complexa problemática fazendo referência ao aumento substancial das médias de qualidades de vida desde o fim do século \textbf{XIX}, argumentando ser consequência direta da revolução científica e sua absorção parcial pela indústria em forma de tecnologia e formas mais eficientes de gerenciamento. 
		
		O cerne do argumento é que se a melhora técnico-científica da produção foi capaz de desfazer inúmeros males históricos, esta também será capaz de resolver os novos males que enfrentamos hoje, a saber a emergência climática e ambiental cada vez mais notória, mesmo que esta tenha sido causada justamente pela maior capacidade produtiva proveniente desta revolução científico-técnica. 

		Queremos, entretanto, entender melhor o papel da tecnologia e da ciência na busca da transição, entendendo inclusive que esta pode ser onerosa, dependendo da situação, às nossas necessidades. O segundo objetivo, além de entender as novas realidades do setor agrícola, é propor intervenções mais adaptadas ao estado atual, atingindo um uso mais consciente da técnica.
	
	\subsection{Escopo do Estudo}

	Em termos das estapas das cadeias de valor da agricultura, diferentemente do setor do petróleo, sabe-se que a maior parte do impacto ambiental negativo vem da produção, isto é, da transformação de matéria prima seja na forma de recurso natural ou não, em bens de consumo ou de capital, e em segundo lugar na etapa da distribuição/logística, isto é, do transporte e armazenamento dos produtos, sendo aquela resposável por (ref) em geral e (ref) na agricultura e esta sendo responsável por (ref) em geral e (ref). Sendo assim focamos nossos estudos nestas duas fases mais críticas. 
	
	Já em termos dos atores sócio-econômicos envolvidos na tomada de decisão, aqueles que mais influenciaram e atualmente têm capacidade de influenciar mudanças operacionais são o Estado e as empresas produtivas. A infraestrutura, lugar fulcral da distribuição, é na maioria das vezes responsabilidade do poder público do Estado enquanto a operacionalização da produção fica a cargo da empresas, algumas vezes amparadas por apoio governamental como pesquisa e desenvolvimento em insumos, maquinário, construções, ciência etc, através de centros de pesquisa.
	
	Pretendemos produzir inteligência para ambos agentes.
	
	Nos apoiamos na literatura do relativamente recente campo da agricultura de precisão para modelar matemáticamente uma unidade produtiva e com isto basear a criação de modelos estatísticos de uso de recursos e crescimento da colheita.
	
	\subsection{Base Teórica}

		Partimos do princípio da necessidade de entendimento do ciclo de vida total e concreta do produto agrícola, observando a relativa independência das diversas fases da produção, em termos gerais - produção, distribuição, consumo, despejo - e formulamos uma metodologia para tanto modelar o impacto ambiental de duas fases mais críticas - produção e distribuição (ref) - mostrando a relevância do estudo deste setor econômico na discussão acadêmica e pública, ao mesmo tempo em que contribuímos para a mitigação de parte desse impacto através da otimização e utilização de técnicas amplamente utilizadas na indústria.

		A noção de desenvolvimento enquanto crescimento bruto da atividade econômica vem sendo criticada mais frequentemente nos últimos anos à medida que crescem as pressões por mudanças nos modelos produtivos e de consumo. Alguns autores (ref) defendem a possibilidade de um desenvolvimento econômico sem crescimento, isto é, um aumento da capacidade social de satisfação das necessidades dos indivíduos e grupos sem aumento da produção total, ou a manutenção desta capacidade de satisfaçao de necessidades utilizando-se menos recursos como insumos produtivos.
		
		Amparamo-nos também no estudo do efeito sócio-ambiental da tecnologia (ref) para mostrar que 
		
		Em primeiro lugar, levantamos bibliografia sobre o impacto ambiental da logística e armazenamento (ref), muitas vezes considerados como partes da distribuição da produção, e argumentamos que, mesmo sendo menos impactante que o setor de produção energética de combustíveis fósseis, ou da produção agropecuária, esta etapa não pode ser desconsiderada na perspectiva da urgência de atos de mitigação pública ou privada por exibir um alto grau de emissão de gases \textbf{CO2}-equivalentes. 
		
		Este problema é, inclusive, provavelmente subestimado, se levarmos em consideração a qualidade da infraestrutura, baixo nível de investimento e baixo nível técnico científico dos principais atores da logística em níveis nacional, estadual e municipal, além da subestimação da perda. 


	\section{Problema de Pesquisa e Objetivo}

		Como observado, o desenvolvimento técnico-científico oferece maneiras de se resolver problemas postos mas no caso da agricultura, ao resolver o problema do limite populacional, acabou por criar uma outra vertente de problemática, a ambiental. A esta nova vertente, propõe-se novamente mais desenvolvimento técnico-científico sem mencionar alguma mudança estrutural do modelo produtivo. Sem dúvida, avanços significativos podem ser alcançados, dificuldades concretas podem ser mitigadas e até alguns problemas podem ser solucionados se seguirmos a direção do avanço técnico-científico "puro". 
		
		Queremos investigar, entretanto, os limites desta linha de ataque às crises ambientais e econômicas já presentes e que ainda virão. Para tal almejamos estimar a capacidade de diminuição do impacto ambiental negativo e contribuição de aumento do impacto ambiental positivo ao desenvolver tecnologias baseadas em pesquisa de otimização. Isto é, criar soluções na linha de avanço técnico-científico, em particular sob a agricultura de precisão, que se propõe como modelo alternativo ao modelo da agricultura industrial, e avaliar sua capacidade de mitigação do impacto ambiental da agricultura.
		
		Mais precisamente, o objetivo central de estudo é estimar, usando ferramentas de \textit{machine learning} (ML) - Aprendizado de máquina - a redução ou aumento do impacto ambiental negativo das novas formas da tecnologia e modelos produtivos baseados em desenvolvimento científico-técnico no ciclo de vida de produtos agrícolas, principalmente na etapa da produção e da distribuição. 
		
		Para estimar o impacto na produção, usaremos um modelo de crescimento de colheita (ref) e desenvolveremos um algoritmo de otimização da utilização de insumos para a colheita com a intenção de observar os limites sustentáveis de acordo com (ref). Na etapa da distribuição usaremos um modelo de grafo de centros urbanos para calcular a distância média percorrida p

	\subsection{Motivação e Relevância}

		A literatura sobre agricultura de precisão já é vasta (ref), e já se ramificou em uma quantidade enorme de vertentes que abarcam as várias questões operacionais da produção agrícola. Um dos tópicos mais estudados é a utilização de \textit{\textbf{ML}} para identificação de deficiências da colheita, sejam no solo, plantação ou fatores climáticos. O manejo sustentável da unidade produtiva permanece pouco entendido porém com algum avanço como por exemplo pode se encontrar em (ref). 
		
		Naquilo que concerne a utilização de insumos na agricultura que, como vimos, é particularmente problemático como impacto ambiental negativo, vamos desenvolver uma tecnologia, usando redes neurais e otimização combinatória (ref) para apontar o manejo ótimo de insumos agrícolas, baseado em resultados estabelecidos da literatura sobre as relações de troca e influência entre plantação, solo e clima. 
		
		Com isto, esperamos avançar a literatura para a integração dos resultados espalhados e fundamentar uma base teórica para um manejo inteligente da unidade produtiva, com utilização otimizada de insumos. Ao mesmo tempo, podemos investigar uma cota inferior de impacto ambiental por utilização de insumos de modo a perceber melhor o limite desta linha de tecnologias na produção e comparar com os parâmetros considerados necessários e/ou suficientes de acordo com as organizações competentes (ref).
		
		Já na etapa da distribuição, há parca literatura sobre a mudança do padrão de consumo a partir da digitalização dos processos. No caso do aumento da entrega \textit{online} de alimentos (ref), poucos estudos se debruçaram sobre a questão das emissões pelo aumento do ecommerce, 

	
	\subsection{Hipóteses Centrais}
	
		Nossa visão subjacente é que, por questões sociais estruturais a tecnologia e a técnica, por mais avançadas que sejam, não serão capazes, por si só, de atingir os objetivos do desenvolvimento sustentável (ODSs). Não nos debruçaremos sobre a arguição nesta direção, apenas mencionamos o fato de desde o começo dos debates sobre preservação ambiental e mesmo enquanto inúmeros encontros e setores tentam colaborar para esta transição à sustentabilidade, muito pouco foi alcançado de fato em termos amplos mesmo que com um avanço enorme nas tendências tecnológicas. 
		
		Muitas vezes o que acontece de fato é o oposto e uma nova tecnologia é disseminada na sociedade por satisfazer conveniências não tão necessárias, muitas vezes até supérfluas, e ao fazê-lo pioram a situação já alarmante e urgente.
		
		Nosso trabalho então é tentar contribuir para o debate da crítica da "tecnologia por si só", na direção de entender melhor qual seria o papel concreto dos setores acadêmicos e privados de tecnologia para construir conjuntamente a Grande Transição à Sustentabilidade, que consideramos necessária. 
		
		Formulamos então as seguintes hipóteses concretas e testáveis:
		
	\begin{enumerate}
		\item O aumento tendicial do serviço de \textit{delivery} de alimentos e preparações alimentares através de relações empresa serviço-empresa produtora-cliente-entregador, que é parte da tendência de digitalização e terceirização produtivas, aumentou o impacto da etapa logística do setor agrícola, tanto em termos absolutos quanto em termos relativos.
		
		\item As técnicas de otimização da produção propostas pelo modelo de produção da agricultura de precisão podem de fato aliviar o impacto da etapa da produção deste setor diminuindo a utilização de insumos, e.g., água, nutrientes químicos - NPK - e energia, porém não geram mitigação suficiente para atingir os objetivos do desenvolvimento sustentável. 
	\end{enumerate}

	Estas hipóteses, se verificadas, contribuiriam para incrementar a tomada de decisão de agentes sociais, públicos-governamentais ou corporativos.
	
	
	Ademais, a técnica de otimização que desenvolveremos para testar e verificar a segunda hipótese pode ser aplicada concretamente em certas unidades produtivas e contribuir ativamente para redução de consumo de insumos agrícolas. 

	Por fim, mencionamos temas tangenciais que também se devidamente estudados, corroborariam ainda mais com a tese.


	\section{Metodologia}

	
	\subsection{Distribuição}
	
	Além do já levantado, queremos mostrar que o impacto aumentou ainda mais com a tendência do aumento de \textit{delivery} de alimentos e preparações alimentícias, exacerbado pela tendência mercadológica das empresas \textit{startups} de serviços tecnologia , como \textbf{\textit{Ifood}} e \textbf{\textit{Rappi}} no Brasil, que neste setor já se encontram muito bem desenvolvidas e têm ampla utilização por grande parcela da população e aumentam o impacto ambiental da logística pós-mercantilização. Este aumento quase não foi considerado enquanto parte do ciclo de vida e menos ainda estudado, uma vez que a maior parte da literatura sobre delivery de alimentos se dedica a entender os hábitos de consumidor, otimização, impacto econômico, técnico e científico, dentre outros.
	
	Soma-se a esta tendência já presente há muitos anos, o efeito de aumento do \textit{\textbf{ecommerce}} durante e pós-pandemia do \textit{SARS-COV-2}, que intensificou a digitalização e terceirização de processos, incluindo preparação alimentar além de ter sido responsável pela criação de inúmeras \textit{dark kitchens}, isto é, empresas com CNPJ ou não cujo único serviço é a confecção de preparações alimentares para entrega por aplicativos. Estimou-se, mesmo que sem rigor, que quase um terço das lojas encontradas no aplicativo \textit{\textbf{Ifood}} em São Paulo em 2023 fossem \textit{dark kitchens}. (ref)
	%https://orbi.band.uol.com.br/sao-paulo/quase-30-dos-restaurantes-do-ifood-sao-dark-kitchens-revela-estudo-6226#:~:text=Pesquisadores%20da%20Unicamp%20analisaram%20estabelecimentos%20de%20São%20Paulo%2C%20Campinas%20e%20Limeira&text=Aproximadamente%20um%20terço%20dos%20restaurantes,é%20do%20tipo%20dark%20kitchens.
	
	Contribuiremos com aprimoramento da metodologia de previsão de dados usando técnicas mais sofisticadas de modelagem estatística, fortemente utilizadas pelo setor privado de tecnologia no âmbito da ciência de dados e aprendizado de máquina (ref).  
	
	Neste sentido estudamos os fatores que levam ao aumento da provisão de serviços destas empresas para estimar o aumento da emissão de gases \textbf{CO}-2 equivalentes da seguinte maneira.


	\begin{enumerate}
		\item O modelo matemático que usaremos é um grafo direcionado rotulado. Este grafo será obtido a partir de dados rodoviários de cidades do \textit{Open Street Map} (\textit{osm}) extraídos de imagens de satélite. Os pontos do grafo significam localidades geográficas de alta precisão, sejam nas rodovias ou em construções, e os elos entre eles significam a presença de uma rodovia cuja mão é a direção do grafo. O rótulo dos pontos significa propensão a demanda por entrega de alimentos ou preparações alimentares, estimativa feita a partir de população, renda e renda per-capita regional. Já o rótulo dos elos signficam a distância em metros entre os pontos. 
		
		\item Estimamos uma distribuição espacial da demanda social por entrega de preparações alimentares ou alimentos em centros urbanos levando em consideração os indicadores econômicos gerais e nacionais disponibilizados publicamente pelas grandes empresas do ramo e, usando estes indicadores como parâmetro bayesianos priorísticos, usamos regressão polinomial ou linear sobre os fatores de população, pib e pib per capita por município - disponibilizados publicamente pela iniciativa dados abertos do governo federal - achar uma distribuição sócio-espacial desse serviço divulgado a nível nacional.
		
		Esta estimativa de demanda por serviços segue metodologias presentes na literatura para outros serviços, principalmente por agentes governamentais (ref). 
		
		Como temos o objetivo ajudar agentes governamentais na tomada de decisão através de embasamento em dados, estatísticas e estimativas e sabendo que as empresas muitas vezes não são necessariamente obrigadas a disponibilizar dados que seriam suficientes para estimar seu impacto ambiental com alta precisão, partimos do pressuposto que é possível obter essas informações a partir de dados publicamente disponíveis, usufruindo da metodologia da coleta de dados sócio-econômicos já feita pelo Estado e frequentemente utilizada para tomada de decisão. 
		
		\item 	Uma vez de posse da estimativa de demanda social por entrega de preparações alimentares ou de alimentos, usaremos a coleta digital de dados de lojas dos dois principais aplicativos que oferecem este serviço, a saber, \textit{\textbf{Ifood}} e \textit{\textbf{Rappi}}, através de ferramentas de raspagem de dados - \textit{webscraping} - usando bibliotecas com código \textit{open source}, mais uma vez tendo em vista a disponibilidade pública destes dados, mesmo que de forma não sistemática, e de acordo com a \textbf{LGPD}, sendo aberto a utilização deste tipo de ferramenta para este tipo de coleta desde que não se interfira na disponibilização do serviço pela própria empresa.
		
		Pretendemos obter dados de uma amostra aleatória de 30 cidades no país, e 20 cidades em países também escolhidos aleatoriamente. Sendo o processo automático e mostrando-se serem insuficientes os dados obtidos, é possível expandir essa seleção, porém sendo ainda limitados por recursos computacionais e tempo.
		
		É importante selecionar municípios bastante diversificados em termos de renda social, idade, região, nível de desenvolvimento econômico, cultura, bioma e relevo para assegurar que a análise não possua viés metodológico.	
		
		\item Um processo de coleta deste tipo produz dados razoavelmente sistematizados, porém com algumas lacunas a serem preenchidas no processo de tratamento de dados através de referências cruzadas com outros dados de outras cidades, preenchimento por médias, medianas ou modas ou até eliminação de certos dados de acordo com o necessário para a análise.
		
		\item Por fim, utilizaremos dados também publicamente disponíveis do iniciativa aberta \textit{Open Street Map}, com informações de redes rodoviárias com alto nível de granularidade - cerca de $50$ metros - (ref)
		
		\item Relacionaremos agora o dado de estimativa de demanda social por região com os dados de oferta industrial de empresas encontradas no passo $2$ para encontrar a distribuição de relação de oferecimento do serviço para cada restaurante e cada região, levand
		
		\item Por fim, agregamos os dados de entrega por estabelecimento em cada região e calculamos  a soma de todas as distâncias percorridas por entregas de acordo com o modelo. Essa é a nossa estimativa da quilometragem de entrega de alimentos por aplicativo a partir da qual podemos estimar a contribuição absoluta e relativa deste emergente setor usando a emissão média por motocicletas e considerando também a proporção de entregas feitas usando veículos que não emitem, como por exemplo, bicicletas.  
		
		
	\end{enumerate}
	
	\subsubsection{Expectativa}
	
	Esperamos obter uma distribuição geoespacial estimada das emissões de gases do efeito estufa por transportes da entrega de alimentos por aplicativo. A partir desta distribuição estimada, podemos calcular os agregados regionais, municipais, estaduais e nacionais deste setor, com detalhamento suficiente para contribuir com inteligência aos tomadores de decisão que não têm acesso aos dados das empresas do setor.
	
	Por exemplo, uma tendência já bem estabelecida do do planejamento urbano é o aumento gradual da malha cicloviária, muitas vezes em detrimento da malha rodoviária convencional. Nosso estudo pode servir como base para um planejamento ótimo ou uma política de transição cicloviária no escopo intra-urbano. 
	
	Além disso, este estudo contribui criando uma metodologia para o melhor entendimento do fluxo rodoviário urbano e serve como passo na direção do urbanismo inteligente (\textit{smart cities}). 
	
	\subsection{Produção}
	
	A etapa da produção é considerada, por consenso científico, a mais problemática em termos de impacto ambiental negativo dentro do ciclo de vida do produto agrícola. Um dado amplamente difundido ao público é a cifra de que a agropecuária é responsável por $70$\% do consumo de água potável no país, porém há ainda outros fatores problemáticos na produção, principalmente associados ao uso muitas vezes indiscriminado de insumos. Podemos citar a acidificação dos oceanos, deterioração e erosão do solo, contaminação de lençóis freáticos (ref).
	
	Além dos insumos, no Brasil um dos maiores problemas causados pela expansão da fronteira agrícola é o desmatamento para suprir os insumos de solo degradados pela agricultura monocultora e intensiva (ref). A perda de ambientes nativos e biodiversidade ainda aumentam o risco de catástrofe ambiental por eventos que normalmente não seriam tão problemáticos ao eliminar as relações redundantes de sistemas complexos ecológicos (ref). Assim chuvas e secas tendem ser mais frequentes e a serem mais intensas.
	
	Por fim e menos entendido ainda é o impacto da engenharia genética para desenho de cultivares ou até engenharia genética animal com o fim de atingir objetivos industriais privados muitas vezes sem regulamentação. 
	
	Aqui contribuiremos para a mitigação do impacto da produção ao minimizar utilização de insumos usando técnicas de sensoreamento remoto, modelagem sistêmica, otimização e robótica em dados georeferenciados e em forma de séries temporais de unidades produtivas. 
	
	Formularemos um problema de otimização combinatória de alocação de recursos (ref) - insumos - baseado na distribuição da necessidade geoespaciotemporal destes estimada por sensoreamento remoto e veículos aéreos não tripulados - UAV, Drones - captados ao longo do tempo e usando análise geoestatística (ref). 
	
	Como mencionamos, a literatura na área de detecção de deficiências e necessidades da colheita é vasta e continua em crescimento, com técnicas cada vez mais eficazes, confiáveis e efetivas, mas a literatura de otimização de gestão de insumos ainda não é tão bem consolidade. Alguns resultados recentes apontam que, quando otimização do manejo é utilizado em unidades produtivas há algum ganho mesmo que modesto, como por exemplo aumento da produtivade por hectare, diminuição do uso de insumos, diminuição do uso de terra dentre outros (ref).

    \subsubsection{O fim da agricultura}

		Sendo a agricultura a única maneira de produzir alimentos na escala necessária e de maneira sustentável e levando em consideração a atual situação de, por um lado uma quantidade enorme de pessoas em insegurança alimentar grave ou leve ao redor do mundo (), e por outro o uso de recursos agrícolas como terra, insumos e mão de obra sendo usados em agricultura para fins industriais há muitas discussões e disputas em curso sobre o objetivo da atividade agrícola. Devemos continuar permitindo o uso cada vez mais dos recursos não renováveis e seus impactos em colheitas que não produzem alimentos? 
				
		Em (ref), os autores, após um estudo bastante abrangente sobre a questão da necessidade de eliminação da fome, um dos ODS, argumentam que é possível alimentar a população do populacional de 10 bilhões de pessoas, segundo estimativa referência da ONU (ref) em 2050, de maneira sustentável e até regenerativa. Entrentanto, neste levantamento, quase não foi deixado espaço para agricultura para fins industriais, como por exemplo a produção de bioenergia, bioquímicos, papel, têxteis, resinas, óleos, cosméticos, etc...
		
		Sendo assim, pelo critério de segurança, devemos avaliar muito criteriosamente se, quando e quanto devemos destinar destes recursos para preservação de habitats nativos, agricultura para fins de alimentação e agricultura para fins industriais. 
		
		É argumentável, então, que o recurso agrícola usado em agricultura com fim industrial deveria ser minimizado, especialmente o solo e a terra. Nesta direção as técnicas de modelagem e otimização da agricultura de precisão têm obtido sucesso moderado em redução do uso de insumos. 
		
	\subsubsection{Eucalipto}
		
		O eucalipto desponta como um exemplo fundamental dessa questão. Um cultivar cujo cultivo de curto prazo inviabiliza o uso do solo de maneira relativamente acelerada na maioria dos tipos de solo por possuir um efeito alelopático (ref), ao passo que é profusamente cultivado com o fim de produzir celulose pra produção de papel e bioenergia sob a forma de carvão ou lenha, mas chegando até a ser usado na indústria da metalurgia, bioquímicos e cosméticos (ref). Objetivamente, o eucalipto dificilmente deixará de ser produzido, de modo que resta ao agentes da sustentabilidade agrícola encontrar maneiras de mudar a relação histórico-concreta dessa cultura com o ambiente e a sociedade. 
		
		Estamos interessados aqui em reduzir o uso de insumos em colheitas de eucalipto para a produção de carvão vegetal, usando metodologias de sensoreamento remoto, modelagem estatística e otimização.  
		 
		Em termos de consumo de recursos agrícolas, o eucalipto consome menos nitrogênio e fósforo do que colheitas convencionais, porém é mais hidro-intensivo do que a maioria dos cultivares, excetuando apenas algumas colheitas como arroz e algodão. Além disso, a literatura sobre o efeito de micronutrientes, consorciamento do eucalipto com outros cultivares, microbiota do solo, físico-química do solo, técnicas de manejo e até preparações nutricionais sofisticadas é grande, e indicam que estas técnicas podem chegar a ter um efeito bastante considerável no rendimento da colheita. 
		
		Formulamos o seguinte problema: Como podemos utilizar recursos - insumos - de forma ótima para diminuir ao máximo o impacto ambiental negativo da produção de eucalipto em uma dada unidade produtiva? Mais precisamente:
		
		\begin{itemize}
			\item Usar dados disponibilizados pelos pesquisadores na literatura sobre insumos, consorciamento, modelagem solo-colheita-clima em colheitas de eucalipto e seus efeitos, assim como literatura de \textit{machine learning} sobre detecção de deficiências nutricionais, de solo ou outras em outras espécies de plantio para, supondo uma distribuição realista de necessidades de insumos em microrregiões, usar uma rede neural e um algoritmo de otimização combinatória exato para resolver um problema de alocação ótima de recursos com efeitos não lineares.
		\end{itemize} 
		
		Queremos discutir e propor solução a este problema com as ferramentas que mencionamos. Subsequentemente, podemos investigar o efeito máximo que uma tal otimização poderia efetuar em uma determinada colheita e discutir quão suficiente seria uma tal otimização. Assim chegando a um passo na investigação da capacidade da proposta de novo modelo produtivo da agricultura de precisão de sanar a problemática do impacto ambiental da produção agrícola.

	Propomos a seguinte esquematização de passo-a-passo para atingir este objetivo:
	
	\begin{enumerate}
		\item Obter dados públicos de uso da terra, na literatura científica sobre eucalipto e \textit{machine learning} para detecção de necessidades de colheitas, dados disponíveis em bases de dados de imagens de satélite como o \textit{open street map} entre outras e comparar com uso de insumos. Junto desta coleta de dados, juntar dados consolidados da literatura sobre o impacto da temperatura, tipo de solo, umidade, tipo de cultivar, tamanho da unidade produtiva e outros fatores afetam o uso de insumos das unidades produtivas, uma vez que há amplo estudo sobre estes fatores. 
		
		Precisamos aqui também dividir a unidade produtiva em microrregiões interligadas para aumentar a precisão do modelo. Aqui entramos na área da agricultura de precisão. O uso de sensores, sejam remotos ou de proximidade nos oferecem informação com maior granularidade de forma quase contínua, para entender com precisão o estado atual da plantação.
		
		Porém não é possível obter sempre todos as medidas necessárias para estimar com exatidação o que precisamos. Assim se fazer necessário que utilizemos de incerteza sobre as distribuições das medidas no território, isto é, teremos apenas o intervalo de confiança de uma medida em uma microrregião.
		
		\item Treinaremos uma rede neural (ref) com os dados obtidos no primeiro passo, uma rede neural para prever a defazagem de cada um dos insumos, água, NPK e agrotóxico, naquela microrregião ou se a microrregião analisada está suficientemente abastecida ou ainda se já está pronta para a colheita, usando também referências teóricas bem estabelecidas.  
		
		A relação entre dados de sensores e defazagem de nutrientes, presença de espécies invasoras entre outras necessidades é indireta de modo que se torna necessário usar um método de modelagem estatística, isto é, um aproximação de uma função objetivo por uma distribuição probabilística com parâmetros a serem aprendidos usando as novas tecnologias de aprendizado de máquina, mais especificamente as redes neurais.
		
		A pesquisa em estimar essas necessidades da colheita por microrregião usando redes neurais ou outras técnicas mais simples de \textit{machine learning} é vasta, porém ainda é focada em altíssima especificidade de cultivar e necessidade. Quase sempre consegue detectar apenas uma necessidade em um cultivar. Queremos aqui generalizar os resultados já obtidos para que uma rede neural consiga identificar pelo menos muitas necessidades diferentes do eucalipto. 
		
		\item Com base nas estimativas de defazagem de insumos, podemos formular o modelo agora como um problema de otimização combinatória, a saber, de alocação ótima de recursos em grafos aleatórios (ref). O resultado da solução deste problema é uma política de aplicação de insumos que minimiza o uso de insumos relativo a produção, ou em outros termos, aumenta a produtividade dos insumos.
		
		Para este problema de otimização combinatória, usaremos restrições com relação à utilização de insumos e impacto ambiental máximo para que a unidade produtiva seja considerada sustentável advindas das recomendações governamentais ou de organismos relevantes supra-nacionais como \textbf{FAO}.
		
		O grafo serve como modelo da unidade produtiva dividida em microrregiões e as conexões entre estas. A aplicação de um insumo em uma microrregião afeta as microrregiões vizinhas, e parametrizando essa relação podemos descobrir como satisfazer a necessidade local de insumos sem comprometer outras regiões e ainda levar em consideração as limitações econômico-ambietais (ref).
		
		\item Calcularemos então as estatísticas sobre parâmetros econômico-ambientais como lucratividade, mudança da margem de gastos, produtividade por insumo, uso total de insumos, redução do uso de insumos com relação à média, redução do impacto ambiental esperado, etc...
		
		\item Com uma estatística de mínimo e máximo de redução de impactos ambientais como acidificação do oceano, emissões ou sequestro de gases do efeito estufa, contaminação do lençol freático por agrotóxicos, degradação e erosão do solo ao usar este tipo de abordagem de solução da agricultura de precisão, poderemos comparar com a necessidade de redução de como entendido pelos órgãos de defesa do ambiente, nacionais e internacionais.
		
		\item Podemos finalmente avaliar o desempenho da técnica desenvolvida em mitigar a utilização de insumos e comparar com os parâmetros considerados necessários ou suficientes pelos organismos internacionais que estudam e prezam pela sustentabilidade da agricultura.
		
		
		%\item Podemos ainda investigar o efeito de utilização de insumos mais sofisticados como preparações nutricionais .
		
	\end{enumerate}
	
	\subsubsection{Expectativa}
	
	Esperamos mostrar que a abordagem da agricultura de precisão é válida e tem efeitos concretos de redução do impacto ambiental na etapa da produção do setor agrícola sem engendrar sacrifícios econômicos ou sociais ou até criando efeitos positivos nestes âmbitos, e que ainda há muitas oportunidades de desenvolvimento técnico-científico em que investir. 
	
	Além disso esperamos criar método/tecnologia de assitência técnica de manejo sustentável de unidades produtivas agrícolas que é capaz de reduzir o \textit{input} de recursos necessários para a produção sem perdas econômicas. 
	
	\section{Resultados Preliminares}
	
	
	\subsection{Distribuição}
	
	Como a primeira hipótese afirma, a tecnologia por si só pode afastar-nos de objetivos concretos e prazos já determinados de redução de emissões ao criar formas de conveniência e com isso aumentar a capacidade produtiva. Neste caso, a tecnologia em questão, o serviço de \textit{delivery} por aplicativo não tem pretensão de contribuir com os esforços de transição à sustentabilidade. Agentes reguladores precisam então de melhor inteligência e com ferramentas de fácil acesso para tomar decisões e salvaguardar o interesse comum neste tipo de situação.
	
	A literatura indica uma correlação entre disponibilidade de inteligência pública e conclusão de mudanças concretas positivas para a sociedade e ambiente. Essa "inteligência pública", a qual afirmamos que nosso estudo incrementa, é argumentavelmente a matéria prima da atuação governamental eficaz nos esforços da transição sustentável. 
	
	\subsubsection{Levantamento de Dados sobre Distâncias Urbanas}
	
	O \textit{Open Street Map} é uma ferramenta de software disponibilizada de forma \textit{open source}, isto é, código aberto 

	\subsubsection{Levantamento de Dados do Restaurantes e do \textit{Ifood}}
	
	Após busca por fontes de dados sobre o lado da oferta operacional do delivery de alimentos, isto é, dos restaurantes presentes no aplicativo \textit{\textbf{Ifood}}, que domina o mercado de entrega online de alimentos, encontramos uma base de dados coletada em 2021, quando o aplicativo atingiu a marca de $300000$ restaurantes cadastrados. Desde então o aumento de número de restaurantes foi pouco expressivo, tendo chegado a $348000$ (ref). Podemos usar este banco de dados como base de fontes de trajetórias de entrega. 
	
	Estes dados foram coletados usando uma técnica de raspagem de dados publicamente disponíveis com a ferramenta \textit{beautifulsoup} e os resultados foram disponibilizados no \textit{Kaggle}, plataforma colaborativa de ciência de dados muito usada para praticar ou aprender técnicas novas de análise e modelagem de dados ou compartilhar dados (ref).
	
	Por questões de segurança e anonimidade, obfuscamos todos os dados referentes à identidade dos estabelecimentos comerciais.
	
	Cada restaurante possui um ponto georeferenciado com latitude e longitude de alta precisão. Usamos este georeferenciamento para encontrar o ponto do grafo urbano obtido como resultado do levantamento de dados anterior, da plataforma \textit{Open Street Map}, e com isso, poderemos calcular a distância deste restaurante a qualquer ponto da cidade em tempo subquadrático usando o algoritmo de Dijkstra, como veremos a seguir.
	
	Além disso, temos dados do relatório econômico geral da empresa dona do capital do \textit{\textbf{Ifood}}. Por exemplo dados sobre o número geral de entregas e aumento no tempo.
	
	\subsubsection{Levantamento de Dados sobre Renda e População em Cidades e Bairros}
	
	Existem dados disponibilizados pela iniciativa dados abertos do governo federal contendo informações populacionais, de PIB e PIB per Capita dos municípios brasileiros baseados no último censo em 2022. 
	
	\subsubsection{Calculando Distância Percorrida por Entrega}
	
	
	
	\subsection{Produção}
	
	A segunda hipótese fala que, mesmo quando a tecnologia é criada para tentar contribuir, ela é insuficiente para suprir as necessidades sociais de preservação ambiental e pode ainda abrir a possibilidade de aumentar a produção e com isso o dano ambiental em outro ponto da cadeia. Nesse caso, é preciso pensar como as forças sociais podem impedir que essa tendência de "preencher o vácuo" deixado pela redução do uso de insumos.
	
	Apontamos para a seguinte direção de pesquisa: De que a forma central através da qual a tecnologia, a técnica e a ciência podem contribuir para aliviar as questões emergenciais e a construir a Grande Transição à Sustentabilidade é quando ela é usada para \textbf{reduzir} insumos, riscos e perdas, ao invés de ser usada para aumentar a produção. Assim a tecnologia contribuiria verdadeiramente para o desenvolvimento econômico sem necessariamente causar \textbf{crescimento} econômico (ref). 
	
	Como argumentamos, uma redução deste tipo abriria espaço para que agentes econômicos se percebam com a abertura para aumentar a produção de outro ou do mesmo tipo que foi reduzida e a esta tendência precisa contrapor uma força político-social para assegurar os objetivos do desenvolvimento sustentável. 
	
	\subsubsection{Levantamento de Literatura sobre Impacto e Necessidades do Eucalipto}
	
	\subsubsection{Levantamento de Literatura sobre \textit{Machine Learning} em Detecção de Necessidades da Colheita}
	
	\subsubsection{Organização dos Dados }
	
	
	
	\section{Referências}
	
\end{document}