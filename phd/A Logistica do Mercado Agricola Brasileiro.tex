\documentclass[]{article}

%opening
\title{A Logística no Sistema Agroalimentar Brasileiro}
\author{Henrique Teixeira Tyrrell Tavares}

\begin{document}
	
	\maketitle
	
	\begin{abstract}
		
	\end{abstract}
	
	\section{Introdução}

		O transporte dos produtos do setor agrícola tem ''\\
														
														
														
														
															
															
															
														
														
															
															
															
															
															
															
																																																																										.	impacto similar ao da produção porém, infelizmente, ele tende a ser subestimado na maior parte da literatura (ref). Um entendimento mais detalhado e seguro da logística do setor agrícola oferece muitas oportunidades para o desenvolvimento técnico-científico nacional, para o investimento em infraestrutura do setor público, para o desenvolvimento sustentável.
		
		O armazenamento do setor tem também impacto subestimado, 
		
		Os produtos agrícolas tendem a ser facilmente perecíveis, levando em muitos casos a um excesso de perda produtiva e desperdício evitáveis. Costumava-se vincular a perda alimentar à falta de infraestrutura das cadeias de suprimento, e portanto aos países subdesenvolvidos, enquanto que o desperdício estaria vinculado ao super-consumo das classes mais afluentes em particular em restaurantes e eventos, e portanto aos países desenvolvidos. Todavia esta figura já foi refutada (ref), sendo o desperdício de países subdesenvolvidos bem maior do que se estimava antes de análise mais detalhada. 
	
	\section{Mercado Interno, Importação e Exportação}
	
		O sistema agroalimentar brasileiro é fortemente dividido entre importação/exportação e mercado interno. Em termos gerais as importações são dominadas pelo grande varejo, em sua maioria as grandes cadeias de supermercados nacionais, a exportação é dominada pelas grandes empresas do agronegócio focada em exportação de algumas poucas commodities agrícolas, enquanto o mercado interno é abastecido principalmente pela agricultura familiar tradicional. 
	

	\section{Comparação com Outros Setores do Mercado Agrícola}

	\section{Armazenamento}
		O armazenamento é realizado principalmente por lojas do grande monopólios do varejo alimentar.
		
	\subsection{Na Unidade Produtiva}
	
	\subsection{Na Indústria}
	
	\subsection{No Varejo}
	\subsection{No Consumidor}
		
	\section{Transporte}
		Por questões históricas, a malha ferroviária brasileira é fortemente subdesenvolvida e a malha aquaviária ainda está se desenvolvendo, tornando o país dependente do modal rodoviário de transporte. Este é o modal menos eficiente em termos gerais e por meio dele se dá quase todo o transporte nacional de produtos agrícolas. 

Recentemente o setor de entregas de alimentos aumentou consideravelmente, especialmente devido à pandemia do sars-cov-2. Esse novo desenvolvimento mudou a configuração do transporte de alimentos na cidade grande, com impacto ambiental e econômico ainda pouco compreendidos (ref). 

	\subsection{Modais e Submodais}

		Dentro do modal rodoviário, o transporte pode ser realizado de inúmeras maneiras levando a resultados econômicos e ambientais muito variados. A entrega de produtos a domicílio por meio de motocicletas no Rio de Janeiro tem impacto relativo muito maior se comparado ao transporte por caminhões em estradas atravessando estados de menor densidade demográfica. Muitos fatores do transporte interferem no seu impacto, como capacidade de carga, bom estado de manutenção do veículo transportador, densidade de tráfego rodoviário, dentre outros. Há poucos estudos quantitativos sobre o efeito direto e indireto destes fatores sobre o transporte.
		
		A operacionalização do transporte usando diversos submodais é um desafio para a indústria logística.   


	\subsection{Insumos}
	
	\subsection{Para Consumidor Final}
	
	O produto pode chegar ao consumidor final de muitas formas. Temos o seguinte gráfico mostrando um resumo das formas mais preponderantes (ref).
	
	(dash)
	
	O restaurantes recebe os alimentos normalmente por entrega direta por caminhões de feiras, supermercados, mercados e centros de distribuição, onde são processados em preparações alimentares.
	
	Por outro lado, os clientes podem se deslocar ao restaurante ou encomendar a preparação alimentar. A pandemia do novo coronavírus resultou no aumento expressivo do mercado de entregas de alimentos (ref). Este aumento ainda não foi analisado sob o ponto de vista da pegada de carbono. 
	
	
	
	\subsection{Para Armazenamento}
	
	O armazenamento de produtos agrícolas acontece principalmente na própria unidade produtiva, nas grandes lojas do varejo, e centros de distribuição. Além disso 
	
	\subsection{Para Processamento}
	
	\subsection{Para Despejo}
	
	
	\section{O Impacto Ambiental da Logística Agrícola}
	
	\section{}
	
	
	
	
	
	
	\section{Problemas}
	
	Problemas a serem mencionados ou discutidos ou não.
	
	\begin{enumerate}

		\item 
	\end{enumerate}
	
	\section{Perguntas}
	
	Perguntas para orientadores
	
	\begin{enumerate}
		
		\item 
		
	\end{enumerate}
	
\end{document}
