\documentclass[]{article}

%opening
\title{Estado Geral do Mercado Agrícola Brasileiro}
\author{Henrique Teixeira Tyrrell Tavares}

\begin{document}

\maketitle

\begin{abstract}

\end{abstract}

\section{Introdução}

O setor agrícola brasileiro é um dos pilares mais basais da economia nacional, e o tem sido desde o período colonial. O novo modelo de desenvolvimento agrícola, chamado agronegócio, focado na monocultura intensiva e extensiva, propriedade privada, com alto grau de tecnologia  e integração com a indústria de processamento e na exportação é sabidamente de altíssimo impacto ambiental.

De fato apesar do grande foco social e acadêmico atualmente nas questões ambientais do setor da energia, em particular da indústria e consumo do petróleo, os impactos da agricultura são equivalentemente danosos. A perda da biodiversidade, acidificação dos oceanos, emissões de gases do efeito estufa, deterioração dos solos, dentre outros efeitos do setor agrícola não são tão comumente discutidos como prioridade da transição a uma economia sustentável. 

A agricultura também é chave para atingir os Objetivos do Desenvolvimento Sustentável (ref), 

A maioria dos atores (ref) do setor já reconhece a insustentabilidade a longo prazo deste modelo enquanto avaliam e testam modelos alternativos de desenvolvimento agrário. 

O potencial de desenvolvimento técnico-científico deste setor parece ser um dos mais subestimados .



\section{Insumos}

O único insumo usado por toda a produção agrícola é a água. É amplamente difundido o dado de estatísticas do IBGE (ref) mostrando que o setor é responsável por aproximadamente $70\%$ do consumo nacional de água. 

Os insumos de uso mais abrangente através dos diferentes modelos são os fertilizantes químicos a base de nitrogênio, fósforo e potássio,  

Alguns modelos de produção, principalmente do agronegócio tendem a usar de forma intensiva 

\section{Produção}



\section{Transporte}

\section{Processamento}

\section{Armazenamento}

\section{Consumo}

\section{Despejo}


\end{document}
