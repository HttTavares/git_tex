\documentclass[]{article}

%opening
\title{O Estado Atual do Sistema Agroalimentar Brasileiro}
\author{Henrique Teixeira Tyrrell Tavares}

\begin{document}

\maketitle

\begin{abstract}

\end{abstract}

\section{Introdução}

O setor agrícola brasileiro é um dos pilares mais basais da economia nacional, e o tem sido desde o período colonial. O novo modelo de desenvolvimento agrícola, chamado agronegócio, focado na monocultura intensiva e extensiva, propriedade privada, com alto grau de tecnologia  e integração com a indústria de processamento e na exportação é sabidamente de altíssimo impacto ambiental.

De fato apesar do grande foco social e acadêmico atualmente nas questões ambientais do setor da energia, em particular da indústria e consumo do petróleo, os impactos da agricultura são equivalentemente danosos. A perda da biodiversidade, acidificação dos oceanos, emissões de gases do efeito estufa, deterioração dos solos, dentre outros efeitos do setor agrícola não são tão comumente discutidos como prioridade da transição a uma economia sustentável. 

A agricultura também é chave para atingir os Objetivos do Desenvolvimento Sustentável (ref), 

A maioria dos atores (ref) do setor já reconhece a insustentabilidade a longo prazo deste modelo enquanto avaliam e testam modelos alternativos de desenvolvimento agrário. 

O potencial de desenvolvimento técnico-científico deste setor parece ser um dos mais subestimados.



\section{Insumos}

O único insumo usado por toda a produção agrícola é a água. É amplamente difundido o dado de estatísticas do IBGE (ref) mostrando que o setor é responsável por aproximadamente $70\%$ do consumo nacional de água. 

Os insumos de uso mais abrangente através dos diferentes modelos são os fertilizantes químicos a base de nitrogênio, fósforo e potássio, conhecido no mercado pela sigla NPK. 

Alguns modelos de produção, principalmente do agronegócio tendem a usar insumos de forma intensiva, extensiva, e em quase todas as etapas produtivas. Os resultados são conhecidos. 

Em primeiro lugar, vemos o aumento da dependência das dinâmicas do mercado internacional (ref), como por exemplo a alta de preços do NPK provocada pela guerra na Ucrânia, a alta dos preço internacional do petróleo, dentre outros.

O impacto ambiental também é consideravelmente maior (ref). Podemos citar a acidificação dos oceanos (ref), a exaustão dos recursos do solo por desequilíbrio químico (ref), diminuição da biodiversidade por agrotóxicos (ref).

A utilização de máquinas agrícolas é também fortemente difundida nesse modelo produtivo. 




\section{Produção}

O processo produtivo da agricultura se compraz em distribuir seres vivos do reino \textit{plantae} em algum meio para usar relações biológicas principalmente dos reinos \textit{monera} e \textit{fungi}, físicas e químicas, principalmente com a concentração de nutrientes, de modo a desenvolver os indivíduos-alvo ao estágio de consumabilidade por humanos. 

A agricultura é provavelmente o setor da economia que mais demanda área terrestre para a produção (ref). No Brasil, a quantidade de terras destinada à produção agrícola chega a (ref). 

A maioria dos modelos produtivos desse processo tendem a realizá-lo de maneira simplificada com alto grau de utilização insumos bioquímicos levando o solo a se exaurir em termos de microbiota, macrobiota e concentração de micronutrientes. 

O exemplo mais clássico é a utilização de agrotóxicos na plantação para a remoção de seres vivos, quase sempre do reino \textit{animalia} considerados pestes, intoxicando-os de modo a salvaguardar a produção agrícola. Os agrotóxicos não eliminam apenas as "pestes" mas também a microbiota necessária para a manutenção dos recursos do solo, além das características físico-químicas necessárias para o desenvolvimento das plantas. Ao longo dos anos o solo é degradado causando a necessidade de expansão do modelo para outros territórios.

\subsection{Agricultura Tradicional Familiar}

Este ainda é o modelo produtivo mais comum se considerarmos o número de unidades produtivas ao invés da área utilizada (ref). A agricultura familiar tradicional é a versão atual do agricultura histórica convencional. Uma família, estendida ou não, detém uma extensão de terra para fins produtivos e a cultiva com técnicas aprendidas por tentativa e erro e passadas através das gerações. A utilização de capital é baixa ou muito baixa, assim como a produtividade, e o requerimento de trabalho para produção é alto. 



\subsection{Agronegócio Extrativista}

A produção relativa é muito alta e com baixa agrobiodiversidade. O principal produto deste modelo atualmente no país é a soja e o milho em alternância, principalmente como insumo da produção pecuária nacional ou internacional (ref). 

*grafico* (principais produtos do agronegocio) 

A distribuição de terras segue um padrão uniforme, pautado principalmente por determinações econômicas e históricas de propriedade de famílias latifundiárias tradicionais. 


\subsection{Agricultura de precisão ou 4.0}

Um modelo que ainda luta para se firmar e se mostrar viável em geral, com alguns sucessos locais, é a agricultura de precisão, também chamada de agricultura inteligente ou 4.0. Consiste principalmente em experimentos e utilização de tecnologias emergentes como internet das coisas (IoT), hidroponia, iluminação artificial, engenharia genética, dentre outras para otimizar os processos produtivos, naturais ou não. Requer alto grau de investimento em capital, especialmente em tecnologia, porém utiliza uma quantidade menor de recursos, trabalho e tem uma produtividade elevada. 


\subsection{Agroecologia}

Surgida de uma cooperação entre agentes acadêmicos, comunidades agrícolas tradicionais e entusiastas da sustentabilidade, a agroecologia utiliza conhecimento sobre os processos ambientais e ecológicos naturais para auxiliar a produção agrícola, mantendo a intervenção artificial em um mínimo viável. Tem impacto ambiental baixíssimo, e por vezes pode restaurar ambientes degradados, tem um custo em trabalho alto, e usa bastante capital técnico, porém pouco maquinário e agentes químicos ou biológicos exógenos. Tem produtividade média.




\section{Transporte}

\subsection{Entre produção e }

Por possuir uma alta demanda de terreno para cultivo, as unidades produtivas estão em média a algumas horas por rodovias dos centros urbanos, em regiões chamadas comumente de cinturões verdes. É necessário então transportar a produção até a população alvo consumidora, seja no campo ou na cidade. O transporte de produtos agrícolas começa com o processamento para transporte nas unidades produtivas, é feito quase que em sua totalidade por rodovias até centros de distribuição e varejo, principalmente supermercados.

\section{Processamento}

A maior parte da produção agrícola não é consumida \textit{in natura} (ref). O ministério da saúde subdivide os alimentos em termos da quantidade de seu processamento em quatro partes: 

\begin{enumerate}
	\item \textit{in natura} ou minimamente processados
	\item Alimentos extraídos da categoria 1
	\item Preparações culinárias a partir de alimentos das categorias 1 e 2
	\item Alimentos ultraprocessados.
\end{enumerate}


\section{Armazenamento}

A produção agrícola tende a ser a menos resiliente, e quanto menos processado, a tendência à resiliência é menor ainda. As perdas e desperdícios alimentares são extremenamente comuns e seu impacto ambiental econômico e social no sistema produtivo são enormes. 

\section{Varejo}

Para o mercado interno, o varejo é a parte que se apropria da maior parte do lucro da produção agrícola.

\section{Consumo}

O consumo de produtos agrícolas é um dos mais fundamentais para a sobrevivência da espécie. 

\section{Despejo}

As sobras 


\section{Problemas}

Problemas a serem mencionados ou discutidos ou não.

\begin{enumerate}
	\item História da agricultura
	
	\item História da agricultura no Brasil
	
	\item Relação da Agricultura com outros setores da economia
	
	\item agricultura só para alimentação ou também para indústria
\end{enumerate}

\section{Perguntas}

Perguntas para orientadores

\begin{enumerate}
	\item O esgoto pode ser considerado como despejo da produção agrícola?
	
	\item Entrega de alimentos em cidades (ifood, ubereats) pode ser considerada transporte do mercado agrícola?
	
	\item 
	
\end{enumerate}


\end{document}
